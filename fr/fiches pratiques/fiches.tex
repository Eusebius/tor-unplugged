\documentclass[a4paper,twoside,french]{article}
\usepackage[T1]{fontenc}
\usepackage{graphicx}
\usepackage[french]{babel}
\usepackage{times}
\usepackage{url}
\usepackage[utf8]{inputenc}
\usepackage{amsmath}
\usepackage{latexsym}
\usepackage{xspace}
\usepackage{multicol}
\usepackage{array}
\sloppy

\addtolength{\topmargin}{-2cm}
\addtolength{\textheight}{+3cm}
\addtolength{\oddsidemargin}{-0,5cm}
\setlength{\evensidemargin}{\oddsidemargin}
\addtolength{\textwidth}{+3cm}

%\renewcommand\section{\@startsection {section}{1}{\z@}%
%	{-3.5ex \@plus -1ex \@minus -.2ex}%
%	{2.3ex \ at plus.2ex}%
%	{\reset@font\Large\bfseries}}

\makeatletter
\renewcommand\section{\@startsection
  {section}{1}{1cm}%name, level, indent
  {1.5\baselineskip}%             beforeskip
  {1\baselineskip}%            afterskip
  {\normalfont\Large\bfseries}}% style
  
\renewcommand\subsection{\@startsection
  {subsection}{2}{2cm}%name, level, indent
  {-\baselineskip}%             beforeskip
  {0.3\baselineskip}%            afterskip
  {\normalfont\normalsize\itshape\bfseries}}% style
  
\renewcommand\subsubsection{\@startsection
  {subsubsection}{3}{2.5cm}%name, level, indent
  {-\baselineskip}%             beforeskip
  {0.3\baselineskip}%            afterskip
  {\normalfont\normalsize\itshape\bfseries}}% style
\makeatother


\newcommand{\fiche}[1]{\cleardoublepage \setcounter{section}{0}{\huge \centerline{\textbf{#1}}\centerline{~}}}

%\makeatletter
%\@addtoreset{section}{fiche}
%\makeatother

\begin{document}
  
  \fiche{Découverte de Tor~: fiche \og émetteur\fg}

  \pagestyle{empty}
  \thispagestyle{empty}

  \section{Votre rôle}

  Vous êtes un \textbf{émetteur}. Vous représentez l'ordinateur d'un
  utilisateur cherchant à communiquer avec un serveur web (le
  \textbf{destinataire}) en passant par le réseau Tor. Vous devrez
  pour cela faire transiter vos messages par un circuit Tor comprenant
  un \textbf{n\oe ud d'entrée}, un \textbf{n\oe ud intermédiaire} et
  un \textbf{n\oe ud de sortie}. Le destinataire devra pouvoir vous
  faire parvenir une réponse par le même circuit.
  
  \section{Organisation}

  \begin{multicols}{2}
    Les participants sont répartis en cinq groupes~:
    \begin{itemize}
    \item Les émetteurs (dont vous faites partie)~;
    \item Les n\oe uds d'entrée~;
    \item Les n\oe uds intermédiaires~;
    \item Les n\oe uds de sortie~;
    \item Les destinataires.
    \end{itemize}
    \vfill\columnbreak

    Vous disposez du matériel suivant~:
    \begin{itemize}
    \item Des enveloppes de grande taille~;
    \item Des enveloppes de taille moyenne~;
    \item Des enveloppes de petite taille~;
    \item Des feuilles de papier entrant dans les petites enveloppes~;
    \item Un crayon, une gomme~;
    \item La présente fiche.
    \end{itemize}
  \end{multicols}

  \section{Envoyer un message}

  \subsection{Construire le circuit}

  \begin{enumerate}
  \item Choisir un n\oe ud d'entrée, un n\oe ud intermédiaire, un n\oe
    ud de sortie et un destinataire~;
  \item On suppose que vous partagez un secret (une clé de
    chiffrement) avec chacun des n\oe uds Tor.
  \end{enumerate}

  \subsection{Préparer le message pour le destinataire}

  \begin{enumerate}
  \item Sur une feuille de papier, inscrivez \og Émetteur~:\fg en
    laissant le champ vide, puis \og Destinataire~: \fg en indiquant
    le nom du destinataire que vous avez choisi~;
  \item Inscrivez \og Identifiant du flux TCP~: \og suivi d'un nombre
    (assez grand) choisi au hasard (l'identifiant TCP est une notion
    indépendante de Tor~: elle permet, dans un échange TCP, de faire
    correspondre les messages envoyés et leurs réponses sans risque de
    confusion)~;
  \item Inscrivez un message pour le destinataire, par exemple une
    question à laquelle il devra répondre. Ne donnez pas d'indication
    quant à votre identité.
  \end{enumerate}

  \subsection{Préparer le message pour le n\oe ud de sortie}

  Le message que vous avez écrit sera au final délivré par le n\oe ud
  de sortie de votre circuit. Le message devra lui arriver chiffré,
  et c'est lui qui le déchiffrera et le transmettra. En conséquence~:
  \begin{enumerate}
  \item Placez votre message dans une petite enveloppe~;
  \item Inscrivez sur l'enveloppe \og chiffré avec la clé de~:\fg
    suivi du nom de votre n\oe ud de sortie que vous avez choisi (ceci
    simule le chiffrement avec la clé que vous partagez avec le n\oe
    ud de sortie).
  \end{enumerate}

  \subsection{Préparer le message pour le n\oe ud intermédiaire}

  \begin{enumerate}
  \item Placez la petite enveloppe dans une enveloppe de taille
    moyenne~;
  \item Inscrivez sur l'enveloppe \og chiffré avec la clé de~:\fg suivi du
    nom de votre n\oe ud intermédiaire.
  \end{enumerate}

  \subsection{Préparer le message pour le n\oe ud d'entrée}

  \begin{enumerate}
  \item Placez l'enveloppe de taille moyenne dans une grande
    enveloppe~;
  \item Inscrivez sur l'enveloppe \og chiffré avec la clé de~:\fg suivi du
    nom de votre n\oe ud d'entrée.
  \end{enumerate}

  \subsection{Envoyer le message}

  Donnez la grande enveloppe à votre n\oe ud d'entrée.

  \section{Recevoir une réponse}

  Normalement, le n\oe ud d'entrée devrait vous remettre une grande
  enveloppe avec l'inscription \og chiffré avec la clé de~:\fg suivi de
  votre nom.
  \begin{enumerate}
  \item Ouvrez la grande enveloppe (si elle est chiffrée avec une clé
    que vous connaissez)~;
  \item Ouvrez l'enveloppe moyenne (si elle est chiffrée avec une clé
    que vous connaissez)~;
  \item Ouvrez la petite enveloppe (si elle est chiffrée avec une clé
    que vous connaissez)~;
  \item Prenez connaissance de la réponse à votre message. Vérifiez
    son émetteur et l'identifiant de flux TCP.
  \end{enumerate}

  \section{Conclusions}

  \begin{itemize}
  \item Parmi les divers acteurs avec lesquels vous avez communiqué,
    lesquels connaissent votre identité~?
  \item Parmi les divers acteurs avec lesquels vous avez communiqué,
    lesquels connaissent l'identité du destinataire~?
  \item Parmi les divers acteurs avec lesquels vous avez communiqué,
    lesquels connaissent le contenu du message et de sa réponse~?
  \item Que faudrait-il changer aux manipulations que vous avez
    effectuées si vous souhaitiez mettre en place un chiffrement dit
    \og de bout en bout\fg entre votre destinataire et vous (comme
    dans le cas d'une communication HTTPS)~?
  \end{itemize}
  
  \fiche{Découverte de Tor~: fiche \og n\oe ud d'entrée\fg}

  \section{Votre rôle}

  Vous êtes un \textbf{n\oe ud d'entrée}. Vous représentez un serveur
  participant au projet Tor, destiné à être en contact direct avec les
  utilisateurs (\textbf{émetteurs}) en tant que premier élément du
  circuit qu'ils auront choisi pour joindre le
  \textbf{destinataire}. Votre rôle est de faire transiter les messages
  de l'émetteur au destinataire via un \textbf{n\oe ud intermédiaire}
  et un \textbf{n\oe ud de sortie}. De même, vous serez amené à
  transmettre les réponses du destinataire à l'émetteur par le même
  chemin.
  
  \section{Organisation}

  \begin{multicols}{2}
    Les participants sont répartis en cinq groupes~:
    \begin{itemize}
    \item Les émetteurs~;
    \item Les n\oe uds d'entrée (dont vous faites partie)~;
    \item Les n\oe uds intermédiaires~;
    \item Les n\oe uds de sortie~;
    \item Les destinataires.
    \end{itemize}
    \vfill\columnbreak

    Vous disposez du matériel suivant~:
    \begin{itemize}
    \item Des enveloppes de grande taille (optionnel)~;
    % \item Des enveloppes de taille moyenne~;
    % \item Des enveloppes de petite taille~;
    % \item Des feuilles de papier entrant dans les petites enveloppes~;
    \item Un crayon, une gomme~;
    \item La présente fiche.
    \end{itemize}
  \end{multicols}

  \section{Transmettre un message de l'émetteur}

  Un enveloppe représente le chiffrement utilisé pour protéger un
  message, et seules les personnes possédant la clé de déchiffrement
  peuvent l'ouvrir. On suppose que l'émetteur possède une copie de
  votre clé.

  Un émetteur peut choisir de vous transmettre un message sous la
  forme d'une enveloppe de grande taille.

  \begin{enumerate}
  \item Vérifiez que vous êtes bien identifié comme pouvant ouvrir
    cette enveloppe (il doit y avoir indiqué \og chiffré avec la clé
    de~: \ldots\fg sur l'enveloppe)~;
  \item Ouvrez l'enveloppe et sortez-en une enveloppe de taille
    moyenne (cela simule le déchiffrement de la couche extérieure du
    chiffrement en oignon)~;
  \item Prenez connaissance de l'identité du n\oe ud Tor capable de
    déchiffrer l'enveloppe de taille moyenne~;
  \item Attribuez à cette association entre l'émetteur et le n\oe ud
    désigné un \textbf{numéro de circuit} que vous conserverez dans le
    tableau suivant (à moins que cette association ne vous soit déjà
    connue)~:
    \begin{center}
      \begin{tabular}{|m{3cm}|m{3cm}|m{4cm}|}
        \hline
        Émetteur & N\oe ud intermédiaire &  Numéro de circuit attribué \\
        \hline
        & & \\
        & & \\
        \hline
        & & \\
        & & \\
        \hline
        & & \\
        & & \\
        \hline
        & & \\
        & & \\
        \hline
      \end{tabular}
    \end{center}
  \item Inscrivez sur l'enveloppe de taille moyenne \og numéro de
    circuit~:\fg suivi de ce numéro~;
  \item Transmettez l'enveloppe de taille moyenne au n\oe ud
    intermédiaire désigné~;
  \item Effacez les inscriptions sur l'enveloppe de grande taille et
    conservez-la pour une utilisation ultérieure.
  \end{enumerate}


  \section{Transmettre un message provenant d'un autre n\oe ud Tor}

  Un n\oe ud intermédiaire peut vous transmettre une enveloppe de
  taille moyenne, qui correspond à une réponse en train d'être
  acheminée depuis un destinataire vers un émetteur. Comme vous pouvez
  participer à plusieurs circuits simultanément, le tableau que vous
  avez construit vous aidera à identifier l'émetteur concerné.

  \begin{enumerate}
  \item Vérifiez que le n\oe ud qui vous transmet l'enveloppe et le
    numéro de circuit qui est inscrit dessus correspond bien à l'une
    des lignes de votre tableau~;
  \item Identifiez l'émetteur grâce au tableau~;
  \item Placez l'enveloppe de taille moyenne dans une enveloppe de
    grande taille en inscrivant dessus \og chiffré avec la clé de~:
    \fg suivi de votre nom~;
  \item Transmettez l'enveloppe à l'émetteur.
  \end{enumerate}

  \section{Conclusions}

  \begin{itemize}
  \item En tant que n\oe ud d'entrée, pouvez-vous identifier
    l'émetteur d'un échange de messages~?
  \item En tant que n\oe ud d'entrée, pouvez-vous identifier
    le récepteur d'un échange de messages~?
  \item Pouvez-vous identifier le n\oe ud intermédiaire du circuit~?
  \item Pouvez-vous identifier le n\oe ud de sortie du circuit~?
  \item En tant que n\oe ud d'entrée, connaissez-vous le contenu d'un
    échange de messages~?
  \item Qu'est-ce qui changera pour vous si l'émetteur décide
    d'utiliser, en plus de Tor, un chiffrement dit \og de bout en
    bout\fg entre lui et le destinataire (comme dans le cas d'une
    communication HTTPS)~?
  \item Quel est l'intérêt des numéros de circuits~? Pourrait-on s'en
    passer~? Pourquoi~?
  \end{itemize}
  
  \fiche{Découverte de Tor~: fiche \og n\oe ud intermédiaire\fg}

  \section{Votre rôle}

  Vous êtes un \textbf{n\oe ud intermédiaire}. Vous représentez un
  serveur participant au projet Tor et destiné à participer à un
  circuit Tor. Votre rôle est de passer à un \textbf{n\oe ud de sortie}
  des messages provenant d'un \textbf{n\oe ud d'entrée}. Ces messages
  sont rédigés par un \textbf{émetteur} et adressés à un
  \textbf{destinataire}, tous deux situés hors du circuit Tor. De
  même, vous serez amené à transmettre les réponses du destinataire à
  l'émetteur par le même chemin.
  
  \section{Organisation}

  \begin{multicols}{2}
    Les participants sont répartis en cinq groupes~:
    \begin{itemize}
    \item Les émetteurs~;
    \item Les n\oe uds d'entrée~;
    \item Les n\oe uds intermédiaires (dont vous faites partie)~;
    \item Les n\oe uds de sortie~;
    \item Les destinataires.
    \end{itemize}
    \vfill\columnbreak

    Vous disposez du matériel suivant~:
    \begin{itemize}
    \item Des enveloppes de taille moyenne (optionnel)~;
    % \item Des enveloppes de taille moyenne~;
    % \item Des enveloppes de petite taille~;
    % \item Des feuilles de papier entrant dans les petites enveloppes~;
    \item Un crayon, une gomme~;
    \item La présente fiche.
    \end{itemize}
  \end{multicols}

  \section{Transmettre un message provenant d'un n\oe ud d'entrée}

  Un enveloppe représente le chiffrement utilisé pour protéger un
  message, et seules les personnes possédant la clé de déchiffrement
  peuvent l'ouvrir. On suppose que l'émetteur possède une copie de
  votre clé.

  Un n\oe ud d'entrée peut être amené à vous transmettre un message,
  sous la forme d'une enveloppe de taille moyenne.

  \begin{enumerate}
  \item Vérifiez que vous êtes bien identifié comme pouvant ouvrir
    cette enveloppe (il doit y avoir indiqué \og chiffré avec la clé
    de~: \ldots\fg sur l'enveloppe)~;
  \item Notez le numéro de circuit indiqué sur l'enveloppe et regardez
    dans le tableau ci-dessous si le couple (n\oe ud d'entrée - numéro
    de circuit amont) est déjà présent.\\
    \textbf{Si c'est le cas~:}
    \begin{enumerate}
    \item Ouvrez l'enveloppe et sortez-en une enveloppe de petite
      taille (cela simule le déchiffrement de la couche intermédiaire
      du chiffrement en oignon)~;
    \item Prenez connaissance de l'identité du n\oe ud Tor capable de
      déchiffrer l'enveloppe de petite taille et vérifiez qu'il
      correspond bien à la ligne du tableau que vous avez identifiée~;
    \item Inscrivez sur l'enveloppe de petite taille \og numéro de
      circuit~:\fg suivi du numéro de circuit aval correspondant dans
      le tableau~;
    \item Transmettez l'enveloppe de petite taille au n\oe ud de
      sortie désigné~;
    \item Effacez les inscriptions sur l'enveloppe de taille moyenne
      et conservez-la pour une utilisation ultérieure.
    \end{enumerate}
    \textbf{Si ce n'est pas le cas~:}
    \begin{enumerate}
    \item Ajoutez une entrée dans le tableau en renseignant les deux
      premières colonnes avec ces informations~;
    \item Ouvrez l'enveloppe et sortez-en une enveloppe de petite
      taille (cela simule le déchiffrement de la couche intermédiaire
      du chiffrement en oignon)~;
    \item Prenez connaissance de l'identité du n\oe ud Tor capable de
      déchiffrer l'enveloppe de petite taille~;
    \item Attribuez un numéro de circuit unique et renseignez les deux
      dernières colonnes du tableau~;
    \item Inscrivez sur l'enveloppe de petite taille \og numéro de
      circuit~:\fg suivi du numéro de circuit aval que vous venez de
      choisir~;
    \item Transmettez l'enveloppe de petite taille au n\oe ud de
      sortie désigné~;
    \item Effacez les inscriptions sur l'enveloppe de taille moyenne
      et conservez-la pour une utilisation ultérieure.
    \end{enumerate}
  \end{enumerate}
  \begin{center}
    \begin{tabular}{|m{2.5cm}|m{3.5cm}|m{2.5cm}|m{3.5cm}|}
      \hline
      N\oe ud d'entrée & Numéro de circuit amont &  N\oe ud de sortie & Numéro de circuit aval \\
      \hline
      & & & \\
      & & & \\
      \hline
      & & & \\
      & & & \\
      \hline
      & & & \\
      & & & \\
      \hline
      & & & \\
      & & & \\
      \hline
    \end{tabular}
  \end{center}


  \section{Transmettre un message provenant d'un n\oe ud de sortie}

  Un n\oe ud de sortie peut vous transmettre une enveloppe de
  petite taille, qui correspond à une réponse en train d'être
  acheminée depuis un destinataire vers un émetteur. Comme vous pouvez
  participer à plusieurs circuits simultanément, le tableau que vous
  avez construit vous aidera à identifier l'émetteur concerné.

  \begin{enumerate}
  \item Vérifiez que le n\oe ud de sortie qui vous transmet
    l'enveloppe et le numéro de circuit qui est inscrit dessus
    correspond bien au couple (n\oe ud de sortie, numéro de circuit
    aval) de l'une des lignes de votre tableau~;
  \item Identifiez le n\oe ud d'entrée et le numéro de circuit amont
    correspondants grâce au tableau~;
  \item Placez l'enveloppe de petite taille dans une enveloppe de
    taille moyenne en inscrivant dessus \og chiffré avec la clé de~:
    \fg suivi de votre identité~;
  \item Inscrivez sur l'enveloppe de taille moyenne \og numéro de
    circuit~:\fg suivi du numéro de circuit amont que vous venez
    d'identifier~;
  \item Transmettez l'enveloppe au n\oe ud d'entrée que vous avez
    identifié.
  \end{enumerate}

  \section{Conclusions}

  \begin{itemize}
  \item En tant que n\oe ud intermédiaire, pouvez-vous identifier
    l'émetteur d'un échange de messages~?
  \item En tant que n\oe ud intermédiaire, pouvez-vous identifier
    le destinataire d'un échange de messages~?
  \item Pouvez-vous identifier le n\oe ud d'entrée du circuit~?
  \item Pouvez-vous identifier le n\oe ud de sortie du circuit~?
  \item Que devrions-nous changer dans le protocole pour modifier la
    réponse aux deux questions précédentes~?
  \item En tant que n\oe ud intermédiaire, connaissez-vous le contenu
    d'un échange de messages~?
  \item Qu'est-ce qui changera pour vous si l'émetteur décide
    d'utiliser, en plus de Tor, un chiffrement dit \og de bout en
    bout\fg entre lui et le destinataire (comme dans le cas d'une
    communication HTTPS)~?
  \item Quel est l'intérêt des numéros de circuits~? Pourrait-on s'en
    passer~? Pourquoi~?
  \end{itemize}
  
  \fiche{Découverte de Tor~: fiche \og n\oe ud de sortie\fg}

  \section{Votre rôle}

  Vous êtes un \textbf{n\oe ud de sortie}. Vous représentez un serveur
  participant au projet Tor et destiné à participer à un circuit
  Tor. Votre rôle est de faire sortir un message (passé par un
  \textbf{n\oe ud intermédiaire}) du circuit Tor et de le remettre à
  son \textbf{destinataire}. Ce message provient initialement d'un
  \textbf{émetteur}, qui l'a fait transiter par un \textbf{n\oe ud
    d'entrée}. De même, vous serez amené à transmettre les réponses du
  destinataire à l'émetteur par le même chemin.
  
  \section{Organisation}

  \begin{multicols}{2}
    Les participants sont répartis en cinq groupes~:
    \begin{itemize}
    \item Les émetteurs~;
    \item Les n\oe uds d'entrée~;
    \item Les n\oe uds intermédiaires~;
    \item Les n\oe uds de sortie (dont vous faites partie)~;
    \item Les destinataires.
    \end{itemize}
    \vfill\columnbreak

    Vous disposez du matériel suivant~:
    \begin{itemize}
    \item Des enveloppes de petite taille (optionnel)~;
    % \item Des enveloppes de taille moyenne~;
    % \item Des enveloppes de petite taille~;
    % \item Des feuilles de papier entrant dans les petites enveloppes~;
    \item Un crayon, une gomme~;
    \item La présente fiche.
    \end{itemize}
  \end{multicols}

  \section{Transmettre un message provenant d'un n\oe ud intermédiaire}

  Un enveloppe représente le chiffrement utilisé pour protéger un
  message, et seules les personnes possédant la clé de déchiffrement
  peuvent l'ouvrir. On suppose que l'émetteur possède une copie de
  votre clé.

  Un n\oe ud intermédiaire peut être amené à vous transmettre un
  message, sous la forme d'une enveloppe de petite taille.

  \begin{enumerate}
  \item Vérifiez que vous êtes bien identifié comme pouvant ouvrir
    cette enveloppe (il doit y avoir indiqué \og chiffré avec la clé
    de~: \ldots\fg sur l'enveloppe)~;
  \item Notez le numéro de circuit indiqué sur l'enveloppe et regardez
    dans le tableau ci-dessous si le couple (n\oe ud intermédiaire -
    numéro de circuit) est déjà présent.\\
    \textbf{Si c'est le cas~:}
    \begin{enumerate}
    \item Ouvrez l'enveloppe et sortez-en le message (cela simule le
      déchiffrement de la couche interne du chiffrement en oignon)~;
    \item Prenez connaissance de l'identité du destinataire et
      vérifiez qu'il correspond bien à la ligne du tableau que vous
      avez identifiée~;
    \item Vérifiez que l'identifiant du flux TCP indiqué sur le
      message correspond bien à cette entrée du tableau, sinon
      ajoutez-le (une même ligne peut contenir plusieurs identifiants
      TCP)~;
    \item Renseignez le champ \og émetteur~:\fg du message avec votre
      propre identité~;
    \item Transmettez le message au destinataire~;
    \item Effacez les inscriptions sur l'enveloppe de petite taille et
      conservez-la pour une utilisation ultérieure.
    \end{enumerate}
    \textbf{Si ce n'est pas le cas~:}
    \begin{enumerate}
    \item Ajoutez une entrée dans le tableau en renseignant les deux
      premières colonnes avec ces informations~;
    \item Ouvrez l'enveloppe et sortez-en le message (cela simule le
      déchiffrement de la couche interne du chiffrement en oignon)~;
    \item Prenez connaissance de l'identité du destinataire et de
      l'identifiant de flux TCP du message (l'identifiant TCP est une
      notion indépendante de Tor~: elle permet, dans un échange TCP,
      de faire correspondre les messages envoyés et leurs réponses
      sans risque de confusion)~;
    \item Renseignez les deux dernières colonnes du tableau avec ces
      informations~;
    \item Renseignez le champ \og émetteur~:\fg du message avec votre
      propre identité~;
    \item Transmettez le message au destinataire~;
    \item Effacez les inscriptions sur l'enveloppe de petite taille et
      conservez-la pour une utilisation ultérieure.
    \end{enumerate}
  \end{enumerate}
  \begin{center}
    \begin{tabular}{|m{3cm}|m{3cm}|m{3cm}|m{3cm}|}
      \hline
      N\oe ud intermédiaire & Numéro de circuit &  Destinataire & Identifiant TCP \\
      \hline
      & & & \\
      & & & \\
      \hline
      & & & \\
      & & & \\
      \hline
      & & & \\
      & & & \\
      \hline
      & & & \\
      & & & \\
      \hline
    \end{tabular}
  \end{center}


  \section{Transmettre un message provenant d'un destinataire}

  Un destinataire peut vous faire parvenir une réponse au message que
  vous lui avez transmis. Comme le destinataire et vous pouvez
  participer à plusieurs flux d'informations simultanément, le tableau
  que vous avez construit vous aidera à identifier l'émetteur
  concerné.

  \begin{enumerate}
  \item Vérifiez que le destinataire qui vous transmet l'enveloppe et
    l'identifiant TCP qui est inscrit dessus correspondent bien au couple
    (destinataire, identifiant TCP) de l'une des lignes de votre
    tableau~;
  \item Identifiez le n\oe ud intermédiaire et le numéro de circuit
    correspondants grâce au tableau~;
  \item Placez le message dans une enveloppe de petite taille en
    inscrivant dessus \og chiffré avec la clé de~: \fg suivi de votre
    identité~;
  \item Inscrivez sur l'enveloppe de petite taille \og numéro de
    circuit~:\fg suivi du numéro de circuit que vous venez
    d'identifier~;
  \item Transmettez l'enveloppe au n\oe ud intermédiaire que vous avez
    identifié.
  \end{enumerate}

  \section{Conclusions}

  \begin{itemize}
  \item En tant que n\oe ud de sortie, pouvez-vous identifier
    l'émetteur d'un échange de messages~?
  \item En tant que n\oe ud de sortie, pouvez-vous identifier
    le destinataire d'un échange de messages~?
  \item Pouvez-vous identifier le n\oe ud d'entrée du circuit~?
  \item Pouvez-vous identifier le n\oe ud intermédiaire du circuit~?
  \item En tant que n\oe ud intermédiaire, connaissez-vous le contenu
    d'un échange de messages~?
  \item Qu'est-ce qui changera pour vous si l'émetteur décide
    d'utiliser, en plus de Tor, un chiffrement dit \og de bout en
    bout\fg entre lui et le destinataire (comme dans le cas d'une
    communication HTTPS)~?
  \item Quel est l'intérêt des numéros de circuits et des identifiants
    TCP~? Pourrait-on s'en passer~? Pourquoi~?
  \end{itemize}
  
  \fiche{Découverte de Tor~: fiche \og destinataire\fg}

  \section{Votre rôle}

  Vous êtes le \textbf{destinataire} d'un message, composé par un
  \textbf{émetteur} qui souhaite le faire transiter par un circuit
  Tor. Il le transmettra à un \textbf{n\oe ud d'entrée}, puis à un
  \textbf{n\oe ud intermédiaire}, puis à un \textbf{n\oe ud de
    sortie}, qui vous l'enverra directement. De même, vous serez amené
  à envoyer vos réponses à l'émetteur par le même chemin.
  
  \section{Organisation}

  \begin{multicols}{2}
    Les participants sont répartis en cinq groupes~:
    \begin{itemize}
    \item Les émetteurs~;
    \item Les n\oe uds d'entrée~;
    \item Les n\oe uds intermédiaires~;
    \item Les n\oe uds de sortie~;
    \item Les destinataires (dont vous faites partie).
    \end{itemize}
    \vfill\columnbreak

    Vous disposez du matériel suivant~:
    \begin{itemize}
    %\item Des enveloppes de petite taille (optionnel)~;
    % \item Des enveloppes de taille moyenne~;
    % \item Des enveloppes de petite taille~;
    \item Des feuilles de papier entrant dans les petites enveloppes~;
    \item Un crayon, une gomme~;
    \item La présente fiche.
    \end{itemize}
  \end{multicols}

  \section{Transmettre un message provenant d'un n\oe ud de sortie}

  Un enveloppe représente le chiffrement utilisé pour protéger un
  message, et seules les personnes possédant la clé de chiffrement
  peuvent l'ouvrir. Dans cette version de l'exercice, vous n'aurez pas
  à manipuler d'enveloppe.

  Un n\oe ud de sortie peut être amené à vous transmettre un
  message.

  \begin{enumerate}
  \item Vérifiez que vous êtes bien identifié comme le destinataire de
    ce message, et prenez connaissance de son émetteur~;
  \item Notez l'identifiant TCP indiqué sur l'enveloppe~;
  \item Sur une nouvelle feuille, indiquez \og émetteur~:\fg suivi de
    votre identité, \og destinataire~:\fg suivi de l'identité de
    l'émetteur du message que vous venez de recevoir et \og
    identifiant TCP~:\fg suivi du numéro présent sur le message que
    vous venez de recevoir~;
  \item Écrivez une réponse au message~;
  \item Transmettez votre message au destinataire que vous avez
    identifié, à savoir le n\oe ud de sortie qui vous l'a transmis.
  \end{enumerate}

  \section{Conclusions}

  \begin{itemize}
  \item En tant que destinataire, pouvez-vous identifier
    l'émetteur d'un échange de messages~?
  \item En tant que destinataire, connaissez-vous le contenu du
    message~?
  \item Pouvez-vous identifier le n\oe ud d'entrée du circuit Tor~?
  \item Pouvez-vous identifier le n\oe ud intermédiaire du circuit Tor~?
  \item Pouvez-vous identifier le n\oe ud de sortie du circuit Tor~?
  \item Qu'est-ce qui pourrait vous permettre de savoir que la
    personne qui vous a transmis le message n'est pas celle qui l'a
    rédigée~?
  \item Qu'est-ce qui pourrait vous permettre de savoir que le message
    est passé par un circuit Tor~?
  \item Qu'est-ce qui changera pour vous si l'émetteur décide
    d'utiliser, en plus de Tor, un chiffrement dit \og de bout en
    bout\fg entre lui et vous (comme dans le cas d'une
    communication HTTPS)~?
  \item Quel est l'intérêt de l'identifiant TCP~? Pourrait-on s'en
    passer~? Pourquoi~?
  \end{itemize}

  
		
  %\bibliographystyle{alpha}
  %\bibliography{biblio}
\end{document}


%%% Local Variables:
%%% mode: latex
%%% TeX-master: t
%%% ispell-local-dictionary: "francais"
%%% End: 