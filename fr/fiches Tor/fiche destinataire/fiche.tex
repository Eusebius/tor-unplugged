\documentclass[a4paper,twoside,french]{article}
\usepackage[T1]{fontenc}
\usepackage{graphicx}
\usepackage[french]{babel}
\usepackage{times}
\usepackage{url}
\usepackage[utf8]{inputenc}
\usepackage{amsmath}
\usepackage{latexsym}
\usepackage{xspace}
\usepackage{multicol}
\usepackage{array}
\sloppy

\addtolength{\topmargin}{-2cm}
\addtolength{\textheight}{+3cm}
\addtolength{\oddsidemargin}{-0,5cm}
\setlength{\evensidemargin}{\oddsidemargin}
\addtolength{\textwidth}{+3cm}

%\renewcommand\section{\@startsection {section}{1}{\z@}%
%	{-3.5ex \@plus -1ex \@minus -.2ex}%
%	{2.3ex \ at plus.2ex}%
%	{\reset@font\Large\bfseries}}

\makeatletter
\renewcommand\section{\@startsection
  {section}{1}{1cm}%name, level, indent
  {1.5\baselineskip}%             beforeskip
  {1\baselineskip}%            afterskip
  {\normalfont\Large\bfseries}}% style
  
\renewcommand\subsection{\@startsection
  {subsection}{2}{2cm}%name, level, indent
  {-\baselineskip}%             beforeskip
  {0.3\baselineskip}%            afterskip
  {\normalfont\normalsize\itshape\bfseries}}% style
  
\renewcommand\subsubsection{\@startsection
  {subsubsection}{3}{2.5cm}%name, level, indent
  {-\baselineskip}%             beforeskip
  {0.3\baselineskip}%            afterskip
  {\normalfont\normalsize\itshape\bfseries}}% style
\makeatother



\begin{document}
  
  \title{Découverte de Tor~: fiche \og destinataire\fg}
  %\author{Guillaume Piolle}
  %\date{2010}
  \date{}

  \maketitle
  \pagestyle{empty}
  \thispagestyle{empty}

  \section{Votre rôle}

  Vous êtes le \textbf{destinataire} d'un message, composé par un
  \textbf{émetteur} qui souhaite le faire transiter par un circuit
  Tor. Il le transmettra à un \textbf{n\oe ud d'entrée}, puis à un
  \textbf{n\oe ud intermédiaire}, puis à un \textbf{n\oe ud de
    sortie}, qui vous l'enverra directement. De même, vous serez amené
  à envoyer vos réponses à l'émetteur par le même chemin.
  
  \section{Organisation}

  \begin{multicols}{2}
    Les participants sont répartis en cinq groupes~:
    \begin{itemize}
    \item Les émetteurs~;
    \item Les n\oe uds d'entrée~;
    \item Les n\oe uds intermédiaires~;
    \item Les n\oe uds de sortie~;
    \item Les destinataires (dont vous faites partie).
    \end{itemize}
    \vfill\columnbreak

    Vous disposez du matériel suivant~:
    \begin{itemize}
    %\item Des enveloppes de petite taille (optionnel)~;
    % \item Des enveloppes de taille moyenne~;
    % \item Des enveloppes de petite taille~;
    \item Des feuilles de papier entrant dans les petites enveloppes~;
    \item Un crayon, une gomme~;
    \item La présente fiche.
    \end{itemize}
  \end{multicols}

  \section{Transmettre un message provenant d'un n\oe ud de sortie}

  Un enveloppe représente le chiffrement utilisé pour protéger un
  message, et seules les personnes possédant la clé de chiffrement
  peuvent l'ouvrir. Dans cette version de l'exercice, vous n'aurez pas
  à manipuler d'enveloppe.

  Un n\oe ud de sortie peut être amené à vous transmettre un
  message.

  \begin{enumerate}
  \item Vérifiez que vous êtes bien identifié comme le destinataire de
    ce message, et prenez connaissance de son émetteur~;
  \item Notez l'identifiant TCP indiqué sur l'enveloppe~;
  \item Sur une nouvelle feuille, indiquez \og émetteur~:\fg suivi de
    votre identité, \og destinataire~:\fg suivi de l'identité de
    l'émetteur du message que vous venez de recevoir et \og
    identifiant TCP~:\fg suivi du numéro présent sur le message que
    vous venez de recevoir~;
  \item Écrivez une réponse au message~;
  \item Transmettez votre message au destinataire que vous avez
    identifié, à savoir le n\oe ud de sortie qui vous l'a transmis.
  \end{enumerate}

  \section{Conclusions}

  \begin{itemize}
  \item En tant que destinataire, pouvez-vous identifier
    l'émetteur d'un échange de messages~?
  \item En tant que destinataire, connaissez-vous le contenu du
    message~?
  \item Pouvez-vous identifier le n\oe ud d'entrée du circuit Tor~?
  \item Pouvez-vous identifier le n\oe ud intermédiaire du circuit Tor~?
  \item Pouvez-vous identifier le n\oe ud de sortie du circuit Tor~?
  \item Qu'est-ce qui pourrait vous permettre de savoir que la
    personne qui vous a transmis le message n'est pas celle qui l'a
    rédigée~?
  \item Qu'est-ce qui pourrait vous permettre de savoir que le message
    est passé par un circuit Tor~?
  \item Qu'est-ce qui changera pour vous si l'émetteur décide
    d'utiliser, en plus de Tor, un chiffrement dit \og de bout en
    bout\fg entre lui et vous (comme dans le cas d'une
    communication HTTPS)~?
  \item Quel est l'intérêt de l'identifiant TCP~? Pourrait-on s'en
    passer~? Pourquoi~?
  \end{itemize}

  
		
  %\bibliographystyle{alpha}
  %\bibliography{biblio}
\end{document}


%%% Local Variables:
%%% mode: latex
%%% TeX-master: t
%%% ispell-local-dictionary: "francais"
%%% End: 