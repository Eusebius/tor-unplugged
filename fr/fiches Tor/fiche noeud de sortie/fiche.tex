\documentclass[a4paper,twoside,french]{article}
\usepackage[T1]{fontenc}
\usepackage{graphicx}
\usepackage[french]{babel}
\usepackage{times}
\usepackage{url}
\usepackage[utf8]{inputenc}
\usepackage{amsmath}
\usepackage{latexsym}
\usepackage{xspace}
\usepackage{multicol}
\usepackage{array}
\sloppy

\addtolength{\topmargin}{-2cm}
\addtolength{\textheight}{+3cm}
\addtolength{\oddsidemargin}{-0,5cm}
\setlength{\evensidemargin}{\oddsidemargin}
\addtolength{\textwidth}{+3cm}

%\renewcommand\section{\@startsection {section}{1}{\z@}%
%	{-3.5ex \@plus -1ex \@minus -.2ex}%
%	{2.3ex \ at plus.2ex}%
%	{\reset@font\Large\bfseries}}

\makeatletter
\renewcommand\section{\@startsection
  {section}{1}{1cm}%name, level, indent
  {1.5\baselineskip}%             beforeskip
  {1\baselineskip}%            afterskip
  {\normalfont\Large\bfseries}}% style
  
\renewcommand\subsection{\@startsection
  {subsection}{2}{2cm}%name, level, indent
  {-\baselineskip}%             beforeskip
  {0.3\baselineskip}%            afterskip
  {\normalfont\normalsize\itshape\bfseries}}% style
  
\renewcommand\subsubsection{\@startsection
  {subsubsection}{3}{2.5cm}%name, level, indent
  {-\baselineskip}%             beforeskip
  {0.3\baselineskip}%            afterskip
  {\normalfont\normalsize\itshape\bfseries}}% style
\makeatother



\begin{document}
  
  \title{Découverte de Tor~: fiche \og n\oe ud de sortie\fg}
  %\author{Guillaume Piolle}
  %\date{2010}
  \date{}

  \maketitle
  \pagestyle{empty}
  \thispagestyle{empty}

  \section{Votre rôle}

  Vous êtes un \textbf{n\oe ud de sortie}. Vous représentez un serveur
  participant au projet Tor et destiné à participer à un circuit
  Tor. Votre rôle est de faire sortir un message (passé par un
  \textbf{n\oe ud intermédiaire}) du circuit Tor et de le remettre à
  son \textbf{destinataire}. Ce message provient initialement d'un
  \textbf{émetteur}, qui l'a fait transiter par un \textbf{n\oe ud
    d'entrée}. De même, vous serez amené à transmettre les réponses du
  destinataire à l'émetteur par le même chemin.
  
  \section{Organisation}

  \begin{multicols}{2}
    Les participants sont répartis en cinq groupes~:
    \begin{itemize}
    \item Les émetteurs~;
    \item Les n\oe uds d'entrée~;
    \item Les n\oe uds intermédiaires~;
    \item Les n\oe uds de sortie (dont vous faites partie)~;
    \item Les destinataires.
    \end{itemize}
    \vfill\columnbreak

    Vous disposez du matériel suivant~:
    \begin{itemize}
    \item Des enveloppes de petite taille (optionnel)~;
    % \item Des enveloppes de taille moyenne~;
    % \item Des enveloppes de petite taille~;
    % \item Des feuilles de papier entrant dans les petites enveloppes~;
    \item Un crayon, une gomme~;
    \item La présente fiche.
    \end{itemize}
  \end{multicols}

  \section{Transmettre un message provenant d'un n\oe ud intermédiaire}

  Un enveloppe représente le chiffrement utilisé pour protéger un
  message, et seules les personnes possédant la clé de déchiffrement
  peuvent l'ouvrir. On suppose que l'émetteur possède une copie de
  votre clé.

  Un n\oe ud intermédiaire peut être amené à vous transmettre un
  message, sous la forme d'une enveloppe de petite taille.

  \begin{enumerate}
  \item Vérifiez que vous êtes bien identifié comme pouvant ouvrir
    cette enveloppe (il doit y avoir indiqué \og chiffré avec la clé
    de~: \ldots\fg sur l'enveloppe)~;
  \item Notez le numéro de circuit indiqué sur l'enveloppe et regardez
    dans le tableau ci-dessous si le couple (n\oe ud intermédiaire -
    numéro de circuit) est déjà présent.\\
    \textbf{Si c'est le cas~:}
    \begin{enumerate}
    \item Ouvrez l'enveloppe et sortez-en le message (cela simule le
      déchiffrement de la couche interne du chiffrement en oignon)~;
    \item Prenez connaissance de l'identité du destinataire et
      vérifiez qu'il correspond bien à la ligne du tableau que vous
      avez identifiée~;
    \item Vérifiez que l'identifiant du flux TCP indiqué sur le
      message correspond bien à cette entrée du tableau, sinon
      ajoutez-le (une même ligne peut contenir plusieurs identifiants
      TCP)~;
    \item Renseignez le champ \og émetteur~:\fg du message avec votre
      propre identité~;
    \item Transmettez le message au destinataire~;
    \item Effacez les inscriptions sur l'enveloppe de petite taille et
      conservez-la pour une utilisation ultérieure.
    \end{enumerate}
    \textbf{Si ce n'est pas le cas~:}
    \begin{enumerate}
    \item Ajoutez une entrée dans le tableau en renseignant les deux
      premières colonnes avec ces informations~;
    \item Ouvrez l'enveloppe et sortez-en le message (cela simule le
      déchiffrement de la couche interne du chiffrement en oignon)~;
    \item Prenez connaissance de l'identité du destinataire et de
      l'identifiant de flux TCP du message (l'identifiant TCP est une
      notion indépendante de Tor~: elle permet, dans un échange TCP,
      de faire correspondre les messages envoyés et leurs réponses
      sans risque de confusion)~;
    \item Renseignez les deux dernières colonnes du tableau avec ces
      informations~;
    \item Renseignez le champ \og émetteur~:\fg du message avec votre
      propre identité~;
    \item Transmettez le message au destinataire~;
    \item Effacez les inscriptions sur l'enveloppe de petite taille et
      conservez-la pour une utilisation ultérieure.
    \end{enumerate}
  \end{enumerate}
  \begin{center}
    \begin{tabular}{|m{3cm}|m{3cm}|m{3cm}|m{3cm}|}
      \hline
      N\oe ud intermédiaire & Numéro de circuit &  Destinataire & Identifiant TCP \\
      \hline
      & & & \\
      & & & \\
      \hline
      & & & \\
      & & & \\
      \hline
      & & & \\
      & & & \\
      \hline
      & & & \\
      & & & \\
      \hline
    \end{tabular}
  \end{center}


  \section{Transmettre un message provenant d'un destinataire}

  Un destinataire peut vous faire parvenir une réponse au message que
  vous lui avez transmis. Comme le destinataire et vous pouvez
  participer à plusieurs flux d'informations simultanément, le tableau
  que vous avez construit vous aidera à identifier l'émetteur
  concerné.

  \begin{enumerate}
  \item Vérifiez que le destinataire qui vous transmet l'enveloppe et
    l'identifiant TCP qui est inscrit dessus correspondent bien au couple
    (destinataire, identifiant TCP) de l'une des lignes de votre
    tableau~;
  \item Identifiez le n\oe ud intermédiaire et le numéro de circuit
    correspondants grâce au tableau~;
  \item Placez le message dans une enveloppe de petite taille en
    inscrivant dessus \og chiffré avec la clé de~: \fg suivi de votre
    identité~;
  \item Inscrivez sur l'enveloppe de petite taille \og numéro de
    circuit~:\fg suivi du numéro de circuit que vous venez
    d'identifier~;
  \item Transmettez l'enveloppe au n\oe ud intermédiaire que vous avez
    identifié.
  \end{enumerate}

  \section{Conclusions}

  \begin{itemize}
  \item En tant que n\oe ud de sortie, pouvez-vous identifier
    l'émetteur d'un échange de messages~?
  \item En tant que n\oe ud de sortie, pouvez-vous identifier
    le destinataire d'un échange de messages~?
  \item Pouvez-vous identifier le n\oe ud d'entrée du circuit~?
  \item Pouvez-vous identifier le n\oe ud intermédiaire du circuit~?
  \item En tant que n\oe ud intermédiaire, connaissez-vous le contenu
    d'un échange de messages~?
  \item Qu'est-ce qui changera pour vous si l'émetteur décide
    d'utiliser, en plus de Tor, un chiffrement dit \og de bout en
    bout\fg entre lui et le destinataire (comme dans le cas d'une
    communication HTTPS)~?
  \item Quel est l'intérêt des numéros de circuits et des identifiants
    TCP~? Pourrait-on s'en passer~? Pourquoi~?
  \end{itemize}

  
		
  %\bibliographystyle{alpha}
  %\bibliography{biblio}
\end{document}


%%% Local Variables:
%%% mode: latex
%%% TeX-master: t
%%% ispell-local-dictionary: "francais"
%%% End: 