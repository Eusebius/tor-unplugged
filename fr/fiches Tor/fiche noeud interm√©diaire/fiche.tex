\documentclass[a4paper,twoside,french]{article}
\usepackage[T1]{fontenc}
\usepackage{graphicx}
\usepackage[french]{babel}
\usepackage{times}
\usepackage{url}
\usepackage[utf8]{inputenc}
\usepackage{amsmath}
\usepackage{latexsym}
\usepackage{xspace}
\usepackage{multicol}
\usepackage{array}
\sloppy

\addtolength{\topmargin}{-2cm}
\addtolength{\textheight}{+3cm}
\addtolength{\oddsidemargin}{-0,5cm}
\setlength{\evensidemargin}{\oddsidemargin}
\addtolength{\textwidth}{+3cm}

%\renewcommand\section{\@startsection {section}{1}{\z@}%
%	{-3.5ex \@plus -1ex \@minus -.2ex}%
%	{2.3ex \ at plus.2ex}%
%	{\reset@font\Large\bfseries}}

\makeatletter
\renewcommand\section{\@startsection
  {section}{1}{1cm}%name, level, indent
  {1.5\baselineskip}%             beforeskip
  {1\baselineskip}%            afterskip
  {\normalfont\Large\bfseries}}% style
  
\renewcommand\subsection{\@startsection
  {subsection}{2}{2cm}%name, level, indent
  {-\baselineskip}%             beforeskip
  {0.3\baselineskip}%            afterskip
  {\normalfont\normalsize\itshape\bfseries}}% style
  
\renewcommand\subsubsection{\@startsection
  {subsubsection}{3}{2.5cm}%name, level, indent
  {-\baselineskip}%             beforeskip
  {0.3\baselineskip}%            afterskip
  {\normalfont\normalsize\itshape\bfseries}}% style
\makeatother



\begin{document}
  
  \title{Découverte de Tor~: fiche \og n\oe ud intermédiaire\fg}
  %\author{Guillaume Piolle}
  %\date{2010}
  \date{}

  \maketitle
  \pagestyle{empty}
  \thispagestyle{empty}

  \section{Votre rôle}

  Vous êtes un \textbf{n\oe ud intermédiaire}. Vous représentez un
  serveur participant au projet Tor et destiné à participer à un
  circuit Tor. Votre rôle est de passer à un \textbf{n\oe ud de sortie}
  des messages provenant d'un \textbf{n\oe ud d'entrée}. Ces messages
  sont rédigés par un \textbf{émetteur} et adressés à un
  \textbf{destinataire}, tous deux situés hors du circuit Tor. De
  même, vous serez amené à transmettre les réponses du destinataire à
  l'émetteur par le même chemin.
  
  \section{Organisation}

  \begin{multicols}{2}
    Les participants sont répartis en cinq groupes~:
    \begin{itemize}
    \item Les émetteurs~;
    \item Les n\oe uds d'entrée~;
    \item Les n\oe uds intermédiaires (dont vous faites partie)~;
    \item Les n\oe uds de sortie~;
    \item Les destinataires.
    \end{itemize}
    \vfill\columnbreak

    Vous disposez du matériel suivant~:
    \begin{itemize}
    \item Des enveloppes de taille moyenne (optionnel)~;
    % \item Des enveloppes de taille moyenne~;
    % \item Des enveloppes de petite taille~;
    % \item Des feuilles de papier entrant dans les petites enveloppes~;
    \item Un crayon, une gomme~;
    \item La présente fiche.
    \end{itemize}
  \end{multicols}

  \section{Transmettre un message provenant d'un n\oe ud d'entrée}

  Un enveloppe représente le chiffrement utilisé pour protéger un
  message, et seules les personnes possédant la clé de déchiffrement
  peuvent l'ouvrir. On suppose que l'émetteur possède une copie de
  votre clé.

  Un n\oe ud d'entrée peut être amené à vous transmettre un message,
  sous la forme d'une enveloppe de taille moyenne.

  \begin{enumerate}
  \item Vérifiez que vous êtes bien identifié comme pouvant ouvrir
    cette enveloppe (il doit y avoir indiqué \og chiffré avec la clé
    de~: \ldots\fg sur l'enveloppe)~;
  \item Notez le numéro de circuit indiqué sur l'enveloppe et regardez
    dans le tableau ci-dessous si le couple (n\oe ud d'entrée - numéro
    de circuit amont) est déjà présent.\\
    \textbf{Si c'est le cas~:}
    \begin{enumerate}
    \item Ouvrez l'enveloppe et sortez-en une enveloppe de petite
      taille (cela simule le déchiffrement de la couche intermédiaire
      du chiffrement en oignon)~;
    \item Prenez connaissance de l'identité du n\oe ud Tor capable de
      déchiffrer l'enveloppe de petite taille et vérifiez qu'il
      correspond bien à la ligne du tableau que vous avez identifiée~;
    \item Inscrivez sur l'enveloppe de petite taille \og numéro de
      circuit~:\fg suivi du numéro de circuit aval correspondant dans
      le tableau~;
    \item Transmettez l'enveloppe de petite taille au n\oe ud de
      sortie désigné~;
    \item Effacez les inscriptions sur l'enveloppe de taille moyenne
      et conservez-la pour une utilisation ultérieure.
    \end{enumerate}
    \textbf{Si ce n'est pas le cas~:}
    \begin{enumerate}
    \item Ajoutez une entrée dans le tableau en renseignant les deux
      premières colonnes avec ces informations~;
    \item Ouvrez l'enveloppe et sortez-en une enveloppe de petite
      taille (cela simule le déchiffrement de la couche intermédiaire
      du chiffrement en oignon)~;
    \item Prenez connaissance de l'identité du n\oe ud Tor capable de
      déchiffrer l'enveloppe de petite taille~;
    \item Attribuez un numéro de circuit unique et renseignez les deux
      dernières colonnes du tableau~;
    \item Inscrivez sur l'enveloppe de petite taille \og numéro de
      circuit~:\fg suivi du numéro de circuit aval que vous venez de
      choisir~;
    \item Transmettez l'enveloppe de petite taille au n\oe ud de
      sortie désigné~;
    \item Effacez les inscriptions sur l'enveloppe de taille moyenne
      et conservez-la pour une utilisation ultérieure.
    \end{enumerate}
  \end{enumerate}
  \begin{center}
    \begin{tabular}{|m{2.5cm}|m{3.5cm}|m{2.5cm}|m{3.5cm}|}
      \hline
      N\oe ud d'entrée & Numéro de circuit amont &  N\oe ud de sortie & Numéro de circuit aval \\
      \hline
      & & & \\
      & & & \\
      \hline
      & & & \\
      & & & \\
      \hline
      & & & \\
      & & & \\
      \hline
      & & & \\
      & & & \\
      \hline
    \end{tabular}
  \end{center}


  \section{Transmettre un message provenant d'un n\oe ud de sortie}

  Un n\oe ud de sortie peut vous transmettre une enveloppe de
  petite taille, qui correspond à une réponse en train d'être
  acheminée depuis un destinataire vers un émetteur. Comme vous pouvez
  participer à plusieurs circuits simultanément, le tableau que vous
  avez construit vous aidera à identifier l'émetteur concerné.

  \begin{enumerate}
  \item Vérifiez que le n\oe ud de sortie qui vous transmet
    l'enveloppe et le numéro de circuit qui est inscrit dessus
    correspond bien au couple (n\oe ud de sortie, numéro de circuit
    aval) de l'une des lignes de votre tableau~;
  \item Identifiez le n\oe ud d'entrée et le numéro de circuit amont
    correspondants grâce au tableau~;
  \item Placez l'enveloppe de petite taille dans une enveloppe de
    taille moyenne en inscrivant dessus \og chiffré avec la clé de~:
    \fg suivi de votre identité~;
  \item Inscrivez sur l'enveloppe de taille moyenne \og numéro de
    circuit~:\fg suivi du numéro de circuit amont que vous venez
    d'identifier~;
  \item Transmettez l'enveloppe au n\oe ud d'entrée que vous avez
    identifié.
  \end{enumerate}

  \section{Conclusions}

  \begin{itemize}
  \item En tant que n\oe ud intermédiaire, pouvez-vous identifier
    l'émetteur d'un échange de messages~?
  \item En tant que n\oe ud intermédiaire, pouvez-vous identifier
    le destinataire d'un échange de messages~?
  \item Pouvez-vous identifier le n\oe ud d'entrée du circuit~?
  \item Pouvez-vous identifier le n\oe ud de sortie du circuit~?
  \item Que devrions-nous changer dans le protocole pour modifier la
    réponse aux deux questions précédentes~?
  \item En tant que n\oe ud intermédiaire, connaissez-vous le contenu
    d'un échange de messages~?
  \item Qu'est-ce qui changera pour vous si l'émetteur décide
    d'utiliser, en plus de Tor, un chiffrement dit \og de bout en
    bout\fg entre lui et le destinataire (comme dans le cas d'une
    communication HTTPS)~?
  \item Quel est l'intérêt des numéros de circuits~? Pourrait-on s'en
    passer~? Pourquoi~?
  \end{itemize}

  
		
  %\bibliographystyle{alpha}
  %\bibliography{biblio}
\end{document}


%%% Local Variables:
%%% mode: latex
%%% TeX-master: t
%%% ispell-local-dictionary: "francais"
%%% End: 