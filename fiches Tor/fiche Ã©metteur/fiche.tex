\documentclass[a4paper,twoside,french]{article}
\usepackage[T1]{fontenc}
\usepackage{graphicx}
\usepackage[french]{babel}
\usepackage{times}
\usepackage{url}
\usepackage[utf8]{inputenc}
\usepackage{amsmath}
\usepackage{latexsym}
\usepackage{xspace}
\usepackage{multicol}
\sloppy

\addtolength{\topmargin}{-2cm}
\addtolength{\textheight}{+3cm}
\addtolength{\oddsidemargin}{-0,5cm}
\setlength{\evensidemargin}{\oddsidemargin}
\addtolength{\textwidth}{+3cm}

%\renewcommand\section{\@startsection {section}{1}{\z@}%
%	{-3.5ex \@plus -1ex \@minus -.2ex}%
%	{2.3ex \ at plus.2ex}%
%	{\reset@font\Large\bfseries}}

\makeatletter
\renewcommand\section{\@startsection
  {section}{1}{1cm}%name, level, indent
  {1.5\baselineskip}%             beforeskip
  {1\baselineskip}%            afterskip
  {\normalfont\Large\bfseries}}% style
  
\renewcommand\subsection{\@startsection
  {subsection}{2}{2cm}%name, level, indent
  {-\baselineskip}%             beforeskip
  {0.3\baselineskip}%            afterskip
  {\normalfont\normalsize\itshape\bfseries}}% style
  
\renewcommand\subsubsection{\@startsection
  {subsubsection}{3}{2.5cm}%name, level, indent
  {-\baselineskip}%             beforeskip
  {0.3\baselineskip}%            afterskip
  {\normalfont\normalsize\itshape\bfseries}}% style
\makeatother



\begin{document}
  
  \title{Découverte de Tor~: fiche \og émetteur\fg}
  %\author{Guillaume Piolle}
  %\date{2010}
  \date{}

  \maketitle
  \pagestyle{empty}
  \thispagestyle{empty}

  \section{Votre rôle}

  Vous êtes un \textbf{émetteur}. Vous représentez l'ordinateur d'un
  utilisateur cherchant à communiquer avec un serveur web (le
  \textbf{destinataire}) en passant par le réseau Tor. Vous devrez
  pour cela faire transiter vos messages par un circuit Tor comprenant
  un \textbf{n\oe ud d'entrée}, un \textbf{n\oe ud intermédiaire} et
  un \textbf{n\oe ud de sortie}. Le destinataire devra pouvoir vous
  faire parvenir une réponse par le même circuit.
  
  \section{Organisation}

  \begin{multicols}{2}
    Les participants sont répartis en cinq groupes~:
    \begin{itemize}
    \item Les émetteurs (dont vous faites partie)~;
    \item Les n\oe uds d'entrée~;
    \item Les n\oe uds intermédiaires~;
    \item Les n\oe uds de sortie~;
    \item Les destinataires.
    \end{itemize}
    \vfill\columnbreak

    Vous disposez du matériel suivant~:
    \begin{itemize}
    \item Des enveloppes de grande taille~;
    \item Des enveloppes de taille moyenne~;
    \item Des enveloppes de petite taille~;
    \item Des feuilles de papier entrant dans les petites enveloppes~;
    \item Un crayon, une gomme~;
    \item La présente fiche.
    \end{itemize}
  \end{multicols}

  \section{Envoyer un message}

  \subsection{Construire le circuit}

  \begin{enumerate}
  \item Choisir un n\oe ud d'entrée, un n\oe ud intermédiaire, un n\oe
    ud de sortie et un destinataire~;
  \item On suppose que vous partagez un secret (une clé de
    chiffrement) avec chacun des n\oe uds Tor.
  \end{enumerate}

  \subsection{Préparer le message pour le destinataire}

  \begin{enumerate}
  \item Sur une feuille de papier, inscrivez \og Émetteur~:\fg en
    laissant le champ vide, puis \og Destinataire~: \fg en indiquant
    le nom du destinataire que vous avez choisi~;
  \item Inscrivez \og Identifiant du flux TCP~: \og suivi d'un nombre
    (assez grand) choisi au hasard (l'identifiant TCP est une notion
    indépendante de Tor~: elle permet, dans un échange TCP, de faire
    correspondre les messages envoyés et leurs réponses sans risque de
    confusion)~;
  \item Inscrivez un message pour le destinataire, par exemple une
    question à laquelle il devra répondre. Ne donnez pas d'indication
    quant à votre identité.
  \end{enumerate}

  \subsection{Préparer le message pour le n\oe ud de sortie}

  Le message que vous avez écrit sera au final délivré par le n\oe ud
  de sortie de votre circuit. Le message devra lui arriver chiffré,
  et c'est lui qui le déchiffrera et le transmettra. En conséquence~:
  \begin{enumerate}
  \item Placez votre message dans une petite enveloppe~;
  \item Inscrivez sur l'enveloppe \og chiffré avec la clé de~:\fg
    suivi du nom de votre n\oe ud de sortie que vous avez choisi (ceci
    simule le chiffrement avec la clé que vous partagez avec le n\oe
    ud de sortie).
  \end{enumerate}

  \subsection{Préparer le message pour le n\oe ud intermédiaire}

  \begin{enumerate}
  \item Placez la petite enveloppe dans une enveloppe de taille
    moyenne~;
  \item Inscrivez sur l'enveloppe \og chiffré avec la clé de~:\fg suivi du
    nom de votre n\oe ud intermédiaire.
  \end{enumerate}

  \subsection{Préparer le message pour le n\oe ud d'entrée}

  \begin{enumerate}
  \item Placez l'enveloppe de taille moyenne dans une grande
    enveloppe~;
  \item Inscrivez sur l'enveloppe \og chiffré avec la clé de~:\fg suivi du
    nom de votre n\oe ud d'entrée.
  \end{enumerate}

  \subsection{Envoyer le message}

  Donnez la grande enveloppe à votre n\oe ud d'entrée.

  \section{Recevoir une réponse}

  Normalement, le n\oe ud d'entrée devrait vous remettre une grande
  enveloppe avec l'inscription \og chiffré avec la clé de~:\fg suivi de
  votre nom.
  \begin{enumerate}
  \item Ouvrez la grande enveloppe (si elle est chiffrée avec une clé
    que vous connaissez)~;
  \item Ouvrez l'enveloppe moyenne (si elle est chiffrée avec une clé
    que vous connaissez)~;
  \item Ouvrez la petite enveloppe (si elle est chiffrée avec une clé
    que vous connaissez)~;
  \item Prenez connaissance de la réponse à votre message. Vérifiez
    son émetteur et l'identifiant de flux TCP.
  \end{enumerate}

  \section{Conclusions}

  \begin{itemize}
  \item Parmi les divers acteurs avec lesquels vous avez communiqué,
    lesquels connaissent votre identité~?
  \item Parmi les divers acteurs avec lesquels vous avez communiqué,
    lesquels connaissent l'identité du destinataire~?
  \item Parmi les divers acteurs avec lesquels vous avez communiqué,
    lesquels connaissent le contenu du message et de sa réponse~?
  \item Que faudrait-il changer aux manipulations que vous avez
    effectuées si vous souhaitiez mettre en place un chiffrement dit
    \og de bout en bout\fg entre votre destinataire et vous (comme
    dans le cas d'une communication HTTPS)~?
  \end{itemize}

  
		
  %\bibliographystyle{alpha}
  %\bibliography{biblio}
\end{document}


%%% Local Variables:
%%% mode: latex
%%% TeX-master: t
%%% ispell-local-dictionary: "francais"
%%% End: 