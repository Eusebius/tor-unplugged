\documentclass[a4paper,twoside,french]{article}
\usepackage[T1]{fontenc}
\usepackage{graphicx}
\usepackage[french]{babel}
\usepackage{times}
\usepackage{url}
\usepackage[utf8]{inputenc}
\usepackage{amsmath}
\usepackage{latexsym}
\usepackage{xspace}
\usepackage{multicol}
\usepackage{array}
\sloppy

\addtolength{\topmargin}{-2cm}
\addtolength{\textheight}{+3cm}
\addtolength{\oddsidemargin}{-0,5cm}
\setlength{\evensidemargin}{\oddsidemargin}
\addtolength{\textwidth}{+3cm}

%\renewcommand\section{\@startsection {section}{1}{\z@}%
%	{-3.5ex \@plus -1ex \@minus -.2ex}%
%	{2.3ex \ at plus.2ex}%
%	{\reset@font\Large\bfseries}}

\makeatletter
\renewcommand\section{\@startsection
  {section}{1}{1cm}%name, level, indent
  {1.5\baselineskip}%             beforeskip
  {1\baselineskip}%            afterskip
  {\normalfont\Large\bfseries}}% style
  
\renewcommand\subsection{\@startsection
  {subsection}{2}{2cm}%name, level, indent
  {-\baselineskip}%             beforeskip
  {0.3\baselineskip}%            afterskip
  {\normalfont\normalsize\itshape\bfseries}}% style
  
\renewcommand\subsubsection{\@startsection
  {subsubsection}{3}{2.5cm}%name, level, indent
  {-\baselineskip}%             beforeskip
  {0.3\baselineskip}%            afterskip
  {\normalfont\normalsize\itshape\bfseries}}% style
\makeatother



\begin{document}
  
  \title{Découverte de Tor~: fiche \og n\oe ud d'entrée\fg}
  %\author{Guillaume Piolle}
  %\date{2010}
  \date{}

  \maketitle
  \pagestyle{empty}
  \thispagestyle{empty}

  \section{Votre rôle}

  Vous êtes un \textbf{n\oe ud d'entrée}. Vous représentez un serveur
  participant au projet Tor, destiné à être en contact direct avec les
  utilisateurs (\textbf{émetteurs}) en tant que premier élément du
  circuit qu'ils auront choisi pour joindre le
  \textbf{destinataire}. Votre rôle est de faire transiter les messages
  de l'émetteur au destinataire via un \textbf{n\oe ud intermédiaire}
  et un \textbf{n\oe ud de sortie}. De même, vous serez amené à
  transmettre les réponses du destinataire à l'émetteur par le même
  chemin.
  
  \section{Organisation}

  \begin{multicols}{2}
    Les participants sont répartis en cinq groupes~:
    \begin{itemize}
    \item Les émetteurs~;
    \item Les n\oe uds d'entrée (dont vous faites partie)~;
    \item Les n\oe uds intermédiaires~;
    \item Les n\oe uds de sortie~;
    \item Les destinataires.
    \end{itemize}
    \vfill\columnbreak

    Vous disposez du matériel suivant~:
    \begin{itemize}
    \item Des enveloppes de grande taille (optionnel)~;
    % \item Des enveloppes de taille moyenne~;
    % \item Des enveloppes de petite taille~;
    % \item Des feuilles de papier entrant dans les petites enveloppes~;
    \item Un crayon, une gomme~;
    \item La présente fiche.
    \end{itemize}
  \end{multicols}

  \section{Transmettre un message de l'émetteur}

  Un enveloppe représente le chiffrement utilisé pour protéger un
  message, et seules les personnes possédant la clé de déchiffrement
  peuvent l'ouvrir. On suppose que l'émetteur possède une copie de
  votre clé.

  Un émetteur peut choisir de vous transmettre un message sous la
  forme d'une enveloppe de grande taille.

  \begin{enumerate}
  \item Vérifiez que vous êtes bien identifié comme pouvant ouvrir
    cette enveloppe (il doit y avoir indiqué \og chiffré avec la clé
    de~: \ldots\fg sur l'enveloppe)~;
  \item Ouvrez l'enveloppe et sortez-en une enveloppe de taille
    moyenne (cela simule le déchiffrement de la couche extérieure du
    chiffrement en oignon)~;
  \item Prenez connaissance de l'identité du n\oe ud Tor capable de
    déchiffrer l'enveloppe de taille moyenne~;
  \item Attribuez à cette association entre l'émetteur et le n\oe ud
    désigné un \textbf{numéro de circuit} que vous conserverez dans le
    tableau suivant (à moins que cette association ne vous soit déjà
    connue)~:
    \begin{center}
      \begin{tabular}{|m{3cm}|m{3cm}|m{4cm}|}
        \hline
        Émetteur & N\oe ud intermédiaire &  Numéro de circuit attribué \\
        \hline
        & & \\
        & & \\
        \hline
        & & \\
        & & \\
        \hline
        & & \\
        & & \\
        \hline
        & & \\
        & & \\
        \hline
      \end{tabular}
    \end{center}
  \item Inscrivez sur l'enveloppe de taille moyenne \og numéro de
    circuit~:\fg suivi de ce numéro~;
  \item Transmettez l'enveloppe de taille moyenne au n\oe ud
    intermédiaire désigné~;
  \item Effacez les inscriptions sur l'enveloppe de grande taille et
    conservez-la pour une utilisation ultérieure.
  \end{enumerate}


  \section{Transmettre un message provenant d'un autre n\oe ud Tor}

  Un n\oe ud intermédiaire peut vous transmettre une enveloppe de
  taille moyenne, qui correspond à une réponse en train d'être
  acheminée depuis un destinataire vers un émetteur. Comme vous pouvez
  participer à plusieurs circuits simultanément, le tableau que vous
  avez construit vous aidera à identifier l'émetteur concerné.

  \begin{enumerate}
  \item Vérifiez que le n\oe ud qui vous transmet l'enveloppe et le
    numéro de circuit qui est inscrit dessus correspond bien à l'une
    des lignes de votre tableau~;
  \item Identifiez l'émetteur grâce au tableau~;
  \item Placez l'enveloppe de taille moyenne dans une enveloppe de
    grande taille en inscrivant dessus \og chiffré avec la clé de~:
    \fg suivi de votre nom~;
  \item Transmettez l'enveloppe à l'émetteur.
  \end{enumerate}

  \section{Conclusions}

  \begin{itemize}
  \item En tant que n\oe ud d'entrée, pouvez-vous identifier
    l'émetteur d'un échange de messages~?
  \item En tant que n\oe ud d'entrée, pouvez-vous identifier
    le récepteur d'un échange de messages~?
  \item Pouvez-vous identifier le n\oe ud intermédiaire du circuit~?
  \item Pouvez-vous identifier le n\oe ud de sortie du circuit~?
  \item En tant que n\oe ud d'entrée, connaissez-vous le contenu d'un
    échange de messages~?
  \item Qu'est-ce qui changera pour vous si l'émetteur décide
    d'utiliser, en plus de Tor, un chiffrement dit \og de bout en
    bout\fg entre lui et le destinataire (comme dans le cas d'une
    communication HTTPS)~?
  \item Quel est l'intérêt des numéros de circuits~? Pourrait-on s'en
    passer~? Pourquoi~?
  \end{itemize}

  
		
  %\bibliographystyle{alpha}
  %\bibliography{biblio}
\end{document}


%%% Local Variables:
%%% mode: latex
%%% TeX-master: t
%%% ispell-local-dictionary: "francais"
%%% End: 